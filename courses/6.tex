\newpage
\section{Дифференциальные уравнения}
\subsection{Простейшие типы уравнений первого порядка: уравнения с разделяющимися переменными..}
\subsubsection{Простейшие типы уравнений первого порядка: уравнения с разделяющимися переменными, однородные, линейные, уравнения в полных дифференциалах.}

    %\includepdf[options]{filename}

\subsubsection{Интегрирующий множитель.}

\subsubsection{Уравнение Бернулли или Риккати.}

\subsubsection{ Метод введения параметра для уравнения первого порядка, не разрешенного относительно производной.}

\subsubsection{Методы понижения порядка дифференциальных уравнений. }

\newpage 
\subsection{Линейные дифференциальные уравнения и линейные системы...}
\subsubsection{Линейные дифференциальные уравнения и линейные системы дифференциальных уравнений с постоянными коэффициентами.}

\subsubsection{Формула общего решения линейного однородного уравнения n-го порядка}

\subsubsection{Отыскание решения линейного неоднородного уравнения в случае, когда правая часть уравнения является квазимногочленом.}

\subsubsection{Уравнение Эйлера.}

\newpage 
\subsection{Формула общего решения линейной однородной системы уравнений \dots}
\subsubsection{Формула общего решения линейной однородной системы уравнений в случае простых собственных значений матрицы коэффициентов системы.}

\subsubsection{Формула общего решения линейной однородной системы в случае кратных собственных значений матрицы коэффициентов системы.}

\subsubsection{Отыскание решения линейной неоднородной системы уравнений в случае, когда свободные члены уравнения являются квазимногочленами. }

\newpage 
\subsection{Матричная экспонента и ее использование для получения формулы общего решения \dots}

\subsubsection{Матричная экспонента и ее использование для получения формулы общего решения и решения задачи Коши для линейных однородных и неоднородных систем уравнений.}