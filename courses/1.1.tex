\section{Введение в математический анализ}

\subsection{Предел числовой последовательности. Единственность предела \dots} %Бесконечно малые последовательности и их свойства. Свойства пределов, связанные с неравенствами. Арифметические операции со сходящимися последовательностями. Теорема Вейерштрасса о пределе монотонной ограниченной последовательности. Число e. Теорема Кантора о вложенных отрезках. Бесконечно большие последовательности и их свойства.}

\subsubsection{Предел числовой последовательности}
\begin{definition} (Эпсилон окрестность)
    \[\forall a \, \forall \eps > 0 \: U_\eps(a) = \{x : \rho (x, a) < \eps \}\]
    \[\textbf{} U_\eps (\pm\infty) = (\pm \infty, \pm \frac{1}{\eps})\]
\end{definition}
\begin{definition}(Предел числовой последовательности)
    \[a \in \bar{\R} = \lim_{n \rightarrow +\infty}{a_n} \Leftrightarrow \forall \eps > 0 \, \exists N \in \N \, \forall n > N : a_n \in U_\eps(a)\]
    Если $$a\in\R$$, то последовательность называется сходящейся
\end{definition}


\subsubsection{Единственность предела}
\begin{theorem}(Единственность предела)
    Числовая последовательность может иметь не более чем один предел.
\end{theorem}
\begin{proof}
    Предположим, что $\exists l_1, l_2 \in \R\ |\ \liml_{n \to \infty} x_n = l_1, \liml_{n \to \infty} x_n = l_2$. Тогда:
    $$
        \System{
        \liml_{n \to \infty} x_n = l_1 \lra \forall \eps > 0\ \exists N_1 \in \N\ |\ \forall n > N_1\ |x_n - l_1| < \eps \lra l_1 - \eps < x_n < l_1 + \eps
        \\
        \liml_{n \to \infty} x_n = l_2 \lra \forall \eps > 0\ \exists N_2 \in \N\ |\ \forall n > N_2\ |x_n - l_2| < \eps \lra l_2 - \eps < x_n < l_2 + \eps
        }
    $$
    Рассмотрим $\eps = \frac{l_2 - l_1}{2} > 0$, $\forall n > max(N_1, N_2)$:
    $$
        \System{
        l_1 + \eps = l_1 + \frac{l_2 - l_1}{2} = \frac{l_1 + l_2}{2}
        \\
        l_2 - \eps = l_2 - \frac{l_2 - l_1}{2} = \frac{l_1 + l_2}{2}
        }
    $$
\end{proof}

\begin{theorem} (Предел произведения б.м. и ограниченной последовательностей)
	Если $\{x_n\}_{n = 1}^\infty$ - бесконечно малая, а $\{y_n\}_{n = 1}^\infty$ ограничена, то $\{x_ny_n\}_{n = 1}^\infty$ - бесконечно малая последовательность.
\end{theorem}
\begin{proof}
	$\{y_n\}_{n = 1}^\infty$ - ограниченная $\Ra \exists M > 0\ |\ \forall n \in \N\ |y_n| \le M$
	
	$\{x_n\}_{n = 1}^\infty$ - бесконечно малая $\Ra \forall \eps > 0\ \exists N \in \N\ |\ \forall n > N\ |x_n| < \frac{\eps}{M}$
\end{proof}


\subsubsection{Свойства предела, связанные с неравенствами}
\begin{theorem}(Свойства предела) 

	\begin{enumerate}
		\item (Ограниченность сходящейся последовательности) Если последовательность сходится, то она ограничена.
		
		\item (Отделенность от нуля и сохранение знака) Если последовательность $\{x_n\}_{n = 1}^\infty$ сходится к $l \neq 0$, то $\exists N \in \N \such \forall n > N\ \sgn x_n = \sgn l$ и $|x_n| > \frac{|l|}{2}$
		
		\item (Переход к пределу в неравенстве) Если $\liml_{n \to \infty} x = x_0$, $\liml_{n \to \infty} y = y_0$ и $\exists N \in \N\ |\ n \ge N\ \ x_n \le y_n$, то $x_0 \le y_0$
		
		\item (О промежуточной последовательности) Если $\liml_{n \to \infty} x_n = \liml_{n \to \infty} z_n = l$ и $\exists N \in \N\ |\ \forall n > N\ \ x_n \le y_n \le z_n$, то $\liml_{n \to \infty} y_n = l$
	\end{enumerate}
\end{theorem}

\begin{proof}
	\begin{enumerate}
		\item По условию, $\exists l \in \R\ |\ \forall \eps > 0\ \exists N \in \N\ |\ \forall n > N\ |x_n - l| < \eps$.
		
		Положим $\eps := 1 > 0$. Тогда $\forall n > N\ l - 1 < x_n < l + 1$. Отсюда следует, что
		\begin{align*}
		x_n \le \max(x_1, x_2, \dots, x_N, l + 1) \Ra \{x_n\}_{n = 1}^\infty - \text{ ограничена сверху}
		\\
		x_n \ge \min(x_1, x_2, \dots, x_N, l - 1) \Ra \{x_n\}_{n = 1}^\infty - \text{ ограничена снизу}
		\end{align*}
		
		\item По условию, $\forall \eps > 0\ \exists N \in \N\ |\ \forall n > N\ |x_n - l| < \eps \lra l - \eps < x_n < l + \eps$.
		
		Тогда, рассмотрим $\eps := \frac{|l|}{2} > 0$.
		\begin{align*}
		l > 0 \Ra x_n > l - \eps = \frac{l}{2} > 0
		\\
		l < 0 \Ra x_n < l + \eps = \frac{l}{2} < 0
		\end{align*}
		
		\item От противного. Пусть $x_0 > y_0$. Тогда, по условию:
		\begin{align*}
		\liml_{n \to \infty} x_n = x_0 \lra \forall \eps > 0\ \exists N_1 \in \N\ |\ \forall n > N_1\ x_0 - \eps < x_n < x_0 + \eps
		\\
		\liml_{n \to \infty} y_n = y_0 \lra \forall \eps > 0\ \exists N_2 \in \N\ |\ \forall n > N_2\ y_0 - \eps < y_n < y_0 + \eps
		\end{align*}
		Рассмотрим $\eps := \frac{x_0 - y_0}{2} > 0$, $\forall n > \max(N_1, N_2)$:
		\[
			y_n < y_0 + \eps = \frac{x_0 + y_0}{2} = x_0 - \eps < x_n
		\]
		Получили противоречие.
		
		\item По условию,
		\[
		\System{
			\liml_{n \to \infty} x_n = l \lra \forall \eps > 0\ \exists N_1 \in \N\ |\ \forall n > N_1\ |x_n - l| < \eps
			\\
			\liml_{n \to \infty} z_n = l \lra \forall \eps > 0\ \exists N_2 \in \N\ |\ \forall n > N_2\ |z_n - l| < \eps
		}
		\]
		Отсюда следует: $l - \eps < x_n \le y_n \le z_n < l + \eps \Ra |y_n - l| < \eps$, то есть $\liml_{n \to \infty} y_n = l$
	\end{enumerate}
\end{proof}

\subsubsection{Арифметические операции со сходящимися последовательностями}

Пусть $\liml_{n \to \infty} x_n = x_0$, $\liml_{n \to \infty} y_n = y_0$. Тогда

\begin{enumerate}
	\item $\liml_{n \to \infty} (x_n + y_n) = x_0 + y_0$
	\item $\liml_{n \to \infty} (x_n - y_n) = x_0 - y_0$
	\item $\liml_{n \to \infty} (x_n \cdot y_n) = x_0 \cdot y_0$
	\item Если $\forall n \in \N\ y_n \neq 0$ и $y_0 \neq 0$, то $\liml_{n \to \infty} \frac{x_n}{y_n} = \frac{x_0}{y_0}$
\end{enumerate}

\begin{proof}
\begin{enumerate}
    \item[1-2.] 
    По определению
    \begin{align*}
        \forall \eps > 0\ \exists N_1 \in \N\ |\ \forall n > N_1\ |x_n - x_0| < \frac{\eps}{2}
        \\
        \forall \eps > 0\ \exists N_2 \in \N\ |\ \forall n > N_2\ |y_n - y_0| < \frac{\eps}{2}
    \end{align*}
    Рассмотрим $\forall n > \max(N_1, N_2)$, тогда
    $$
        |(x_n \pm y_n) - (x_0 \pm y_0)| \le |x_n - x_0| + |y_n - y_0| < \frac{\eps}{2} + \frac{\eps}{2} = \eps
    $$
    \item[3.] 
    $\forall \eps > 0\ \exists N_1 \in \N\ |\ \forall n > N_1\ |x_n - x_0| < \frac{\eps}{2}$
    
    Из теоремы выше, $\exists C > 0\ |\ \forall n \in \N\ |x_n| \le C$
    
    $\liml_{n \to \infty} y_n = y_0 \lra \forall \eps > 0\ \exists N_2 \in \N\ |\ \forall n > N_2\ |y_n - y_0| < \frac{\eps}{2C}$
    
    Рассмотрим $\forall n > \max(N_1, N_2)$:
    $$
        |x_n y_n - x_0 y_0| \le |x_n y_n - x_n y_0| + |x_n y_0 - x_0 y_0| = |x_n| \cdot |y_n - y_0| + |y_0| \cdot |x_n - x_0| < C \cdot \frac{\eps}{2C} + \frac{\eps}{2} = \eps
    $$
    \item[4.]
    По условию,
    \begin{align*}
    	&\forall \eps > 0\ \exists N_1 \in \N\ |\ \forall n > N_1\ |x_n - x_0| < \frac{|y_0|}{2} \cdot \frac{\eps}{2}
    	\\
    	&\forall \eps > 0\ \exists N_2 \in \N\ |\ \forall n > N_2\ |y_n - y_0| < \frac{|y_0|^2}{2(|x_0| + 1)} \cdot \frac{\eps}{2}
    \end{align*}
    Так как $\liml_{n \to \infty} y_n = y_0 \neq 0$, то начиная с некоторого номера $|y_n| > \frac{|y_0|}{2}$. Будем считать, что это верно $\forall n > N_2$ (иначе можно \textit{подвинуть} наше значение $N_2$ вправо настолько, что это станет верно).
   	
    Рассмотрим $\forall n > \max(N_1, N_2)$
    \begin{multline*}
        \left|\frac{x_n}{y_n} - \frac{x_0}{y_0}\right| = \left|\frac{x_n y_0 - y_n x_0}{y_n y_0}\right| \le \frac{|x_n y_0 - x_0 y_0|}{|y_n| \cdot |y_0|} + \frac{|x_0 y_0 - y_n x_0|}{|y_n| \cdot |y_0|} =
        \\
        = \frac{|x_n - x_0|}{|y_n|} + \frac{|x_0| \cdot |y_0 - y_n|}{|y_n| \cdot |y_0|} < |x_n - x_0|\cdot \frac{2}{|y_0|} + |y_0 - y_n| \cdot \frac{2|x_0|}{|y_0|^2} <
        \\
        < \frac{\eps}{2} + \frac{\eps}{2} \cdot \frac{|x_0|}{|x_0| + 1} < \eps
    \end{multline*}
\end{enumerate}
\end{proof}

\subsubsection{Теорема Вейерштрасса о пределе монотонной ограниченной последовательности}
\begin{theorem} (Вейерштрасса о монотонных последовательностях)
	Если $\{x_n\}_{n = 1}^\infty$ ограниченная сверху и неубывающая последовательность, то $\exists \liml_{n \ra \infty} x_n = \sup \{x_n\}$. Если же невозрастающая и ограниченная снизу, то $\liml_{n \ra \infty} x_n = \inf \{x_n\}$
\end{theorem}

\begin{proof}
	Приведём доказательство только для ограниченной сверху и неубывающей последовательности.
	\[
		l := \sup \{x_n\} \lra \System{
		&\forall n \in \N\ x_n \le l
		\\
		&\forall \eps > 0\ \exists N \in \N\ l - \eps < x_N \le l
		}
	\]
	Рассмотрим $\forall n > N$. Тогда
	\[
		l + \eps > l \ge x_n \ge x_{n - 1} \ge \dots \ge x_N > l - \eps \Ra |x_n - l| < \eps
	\]
	То есть
	\[
		\forall \eps > 0\ \exists N \in \N \such \forall n > N |x_n - l| < \eps
	\]
	Что и доказывает наше утверждение.
\end{proof}

\subsubsection{Число e}
\begin{theorem} (Число Эйлера)
	Последовательность $\{x_n = \left(1 + \frac{1}{n}\right)^n\}_{n = 1}^\infty$ сходится. Её предел называется числом $e$.
	$$
		e \approx 2,718281828459045\dots
	$$
\end{theorem}

\begin{proof}
	Рассмотрим последовательность $y_n := \left(1 + \frac{1}{n}\right)^{n + 1}$. Докажем, что $y_n$ убывает.
	\begin{multline*}
		\frac{y_{n - 1}}{y_n} = \frac{(1 + \frac{1}{n - 1})^n}{(1 + \frac{1}{n})^{n + 1}} = \left(\frac{\frac{n}{n - 1}}{\frac{n + 1}{n}}\right)^{n+1} \cdot \frac{1}{1 + \frac{1}{n - 1}} = \left(\frac{n^2}{n^2 - 1}\right)^{n+1} \cdot \frac{1}{1 + \frac{1}{n - 1}} = \\
		\left(1 + \frac{1}{n^2 - 1}\right)^{n+1} \cdot \frac{1}{1 + \frac{1}{n - 1}} \ge \left(1 + \frac{n + 1}{n^2 - 1}\right) \cdot \frac{1}{1 + \frac{1}{n - 1}} = \\
		\left(1 + \frac{1}{n - 1}\right) \cdot \frac{1}{1 + \frac{1}{n - 1}} = 1, n > 1
	\end{multline*}
	При этом $\{y_n\}$ - ограниченная снизу последовательность, так как $\forall n \in \N\ y_n \ge 0$
	
	Следовательно, по теореме Вейерштрасса $\{y_n\}$ сходится. Её предел равен $e$.
	
	Покажем, что к тому же пределу сходится и $x_n$:
	$$
		\liml_{n \to \infty} x_n = \liml_{n \to \infty} y_n \cdot \left(1 + \frac{1}{n}\right) = e \cdot (1 + 0) = e
	$$
\end{proof}
\subsubsection{Теорема Кантора о вложенных отрезках}
\begin{definition}
	Последовательность вложенных отрезков - это $\{[a_n; b_n]\}_{n = 1}^\infty$, $a_n \le b_n\ \forall n \in \N$ такая, что $\forall n \in \N\ [a_n; b_n] \supset [a_{n + 1}; b_{n + 1}]$
\end{definition}

\begin{theorem} (Принцип Кантора вложенных отрезков)
	Каждая система вложенных отрезков имеет непустое пересечение, то есть
	\[
		\bigcap\limits_{n = 1}^\infty [a_n; b_n] \neq \emptyset
	\]
\end{theorem}

\begin{proof}
	$[a_n; b_n] \supset [a_{n + 1}; b_{n + 1}] \Ra \left((a_n \le a_{n + 1}) \wedge (b_n \ge b_{n + 1})\right)$
	
	Следовательно, $\{a_n\}$ - неубывающая, а $\{b_n\}$ - невозрастающая
	
	$a_n \le b_n \le b_1$, а $a_1 \le a_n \le b_n$, то есть
	\begin{align*}
		&\exists a = \liml_{n \ra \infty} a_n = \sup \{a_n\}
		\\
		&\exists b = \liml_{n \ra \infty} b_n = \inf \{b_n\}
	\end{align*}
	Так как $\forall n \in \N a_n \le b_n$, то предельный переход даёт неравенство $a \le b$
	
	Ну а учитывая равенства у пределов, получим $a_n \le a \le b \le b_n$, то есть $\forall n \in \N\ [a_n; b_n] \supset [a; b]$, что и доказывает непустоту пересечения.
\end{proof}

\begin{definition}
	Стягивающейся системой отрезков называется система вложенных отрезков, длины которых образуют б.м. последовательность.
\end{definition}

\begin{addition}
	Система стягивающихся отрезков имеет пересечение, состоящее из одной точки.
\end{addition}
\begin{proof}
	$a_n \le a \le b \le b_n \Ra 0 \le b - a \le b_n - a_n \Ra a = b$
\end{proof}

\subsubsection{Бесконечно малые и бесконечно большие последовательности и их свойства}
\begin{definition}
	Последовательность $\{x_n\}_{n = 1}^\infty$ называется \textit{бесконечно малой}, если 
	$$
		\liml_{n \ra \infty} x_n = 0
	$$
\end{definition}

\begin{theorem}(Предел произведения б.м. и ограниченной последовательностей)    
    Если $\{x_n\}_{n = 1}^\infty$ - бесконечно малая, а $\{y_n\}_{n = 1}^\infty$ ограничена, то $\{x_ny_n\}_{n = 1}^\infty$ - бесконечно малая последовательность.
\end{theorem}

\begin{proof}
	$\{y_n\}_{n = 1}^\infty$ - ограниченная $\Ra \exists M > 0\ |\ \forall n \in \N\ |y_n| \le M$
	
	$\{x_n\}_{n = 1}^\infty$ - бесконечно малая $\Ra \forall \eps > 0\ \exists N \in \N\ |\ \forall n > N\ |x_n| < \frac{\eps}{M}$
\end{proof}


\begin{definition}
	Последовательностью $\{x_n\}_{n = 1}^\infty$ называется бесконечно большой, если
	$$
		\liml_{n \ra \infty} x_n = -\infty, +\infty \text{ или } \infty
	$$
\end{definition}

\begin{theorem} (Связь б.м. и б.б. последовательностей)
	Если $x_n \neq 0\ \forall n \in \N$, то $\{x_n\}_{n = 1}^\infty$ - б.м. $\lra \{\frac{1}{x_n}\}_{n = 1}^\infty$ - б.б.
\end{theorem}

\begin{proof}
\begin{enumerate}
	\item $\{x_n\}_{n = 1}^\infty$ - б.м. $\Ra \forall \eps > 0\ \exists N \in \N\ |\ \forall n > N\ |x_n| < \eps$. Отсюда следует, что $\left|\frac{\dse 1}{\dse x_n}\right| > \frac{\dse 1}{\dse \eps} \lra \frac{\dse 1}{\dse x_n} \in U_{\eps}(\infty) \lra \liml_{n \ra \infty} \frac{\dse 1}{\dse x_n} = \infty$
	
	\item $\liml_{n \ra \infty} \frac{1}{x_n} = \infty \lra \forall \eps > 0\ \exists N \in \N\ |\ n > N\ \left|\frac{\dse 1}{\dse x_n}\right| > \frac{\dse 1}{\dse \eps} \Ra 0 < |x_n| < \eps \lra \liml_{n \to \infty} x_n = 0$
\end{enumerate}
\end{proof}

