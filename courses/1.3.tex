\subsection{Предел функции одной переменной. Определения по Гейне и по Коши, \dots } %их эквивалентность. Свойства пределов функции. Различные типы пределов. Критерий Коши существования конечного предела функции. Теорема о замене переменной под знаком предела. Существование односторонних пределов у монотонной функции.

\subsubsection{Предел функции одной переменной. Определения по Гейне и по Коши, их эквивалентность}

\begin{definition} (Предел по Коши)
	$$
		\liml_{x \to a} f(x) = A \lra \forall \eps > 0\ \exists \delta > 0\ |\ \forall x \in \mc{U}_{\delta}(a)\ f(x) \in U_{\eps}(A)
	$$
\end{definition}

\begin{definition} (Предел по Гейне)
	$$
		\liml_{x \to a} f(x) = A \lra \left(\forall \{x_n\} \subset X \bs \{a\}\ \liml_{n \to \infty} x_n = a\right)\ \liml_{n \to \infty} f(x_n) = A
	$$
\end{definition}

\begin{theorem}
	Определения предела функции по Коши и по Гейне эквивалентны.
\end{theorem}

\begin{proof}
	\begin{enumerate}
		\item (К $\Ra$ Г)
		
		Рассмотрим $\forall \{x_n\} \subset X \bs \{a\} \such \liml_{n \to \infty} x_n = a$. По определению предела
		$$
			\forall \delta > 0\ \exists N \in \N \such \forall n > N\ |x_n - a| < \delta
		$$
		Так как $\forall n \in \N\ x_n \in X \bs \{a\}$, то отсюда следует
		$$
			\forall \delta > 0\ \exists N \in \N \such \forall n > N\ x_n \in \mc{U}_{\delta}(a)
		$$
		По условию выполнено утверждение:
		$$
			\liml_{x \to a} f(x) = A \lra \forall \eps > 0\ \exists \delta > 0 \such \forall x \in \mc{U}_{\delta}(a)\ f(x) \in U_{\eps}(A)
		$$
		То есть для любого $\eps > 0$ найдётся $\delta > 0$, для которого верно 2 условия:
		$$
		\System{
			&{\exists N \in \N \such \forall n > N\ x_n \in \mc{U}_{\delta}(a)}
			\\
			&{\forall x \in \mc{U}_{\delta}(a)\ f(x) \in U_{\eps}(A)}
		}
		$$
		В итоге имеем:
		$$
			\forall \eps > 0\ \exists N \in \N \such \forall n > N\ f(x_n) \in U_{\eps}(A) \lra \liml_{n \to \infty} f(x_n) = A
		$$
		
		\item (Г $\Ra$ К)
		
		Докажем от противного, то есть при выполненности определения Гейне неверно определение Коши:
		
		$\exists \eps > 0\ |\ \forall \delta > 0\ \exists x \in \mc{U}_{\delta}(a)\ |\ f(x) \notin U_{\eps}(A)$
		
		Зафиксируем $\eps$ и подставим разные $\delta$:
		\begin{align*}
			&\delta := 1 & &{\exists x_1 \in \mc{U}_{1}(a)} & &{f(x_1) \notin U_{\eps}(A)}
			\\
			&\delta := 1/2 & &{\exists x_2 \in \mc{U}_{1/2}(a)} & &{f(x_2) \notin U_{\eps}(A)}
			\\
			&\dots & &\dots & &\dots
			\\
			&\delta := 1/n & &{\exists x_n \in \mc{U}_{1/n}(a)} & &{f(x_n) \notin U_{\eps}(A)}
			\\
			&\dots & &\dots & &\dots
		\end{align*}
		
		Получили последовательность $\{x_n\}_{n = 1}^\infty \such \forall n \in \N\ x_n \in \mc{U}_{1/n}(a),\ f(x_n) \notin U_{\eps}(A)$
		
		Но при этом, для этой последовательности верно утверждение:
		$$
			\forall \eps > 0\ \exists N \in \N \such \forall n > N\ x_n \in \mc{U}_{\eps}(a) \lra \liml_{n \to \infty} x_n = a
		$$
		А из определения предела по Гейне это будет означать, что
		$$
			\liml_{n \to \infty} f(x_n) = A \lra \forall \eps > 0\ \exists N \in \N \such \forall n > N\ f(x_n) \in U_{\eps} (A)
		$$
		Получили противоречие. (Потому что хотя бы для одного $\eps$, которое мы зафиксировали для последовательности, это выполнено не будет)
	\end{enumerate}
\end{proof}

\subsubsection{Свойства пределов функции}
\begin{enumerate}
    \item (о единственности) Если $\lim_{x \rightarrow a} f(x) = b$ и $\lim_{x \rightarrow a} f(x) = c$, то $b = c$
    \begin{proof}
        Рассмотрим произвольную $\{x_{n}\} \subset E \setminus \{a\}$, $x_{n} \rightarrow a$. По определению Гейне:
        \[f(x_{n}) \rightarrow b \text{ и } f(x_{n}) \rightarrow c\]
        В силу единственности предела последовательности $b = c$.
    \end{proof}
    \item (о пределе по подмножеству) Если $\lim_{x \rightarrow a} f(x) = b$ и $a$ -- предельная точка множества $D \subset E$, то $\lim_{x \rightarrow a} f|_{D}(x) = b$.
    \begin{proof}
        Рассмотрим $\{x_{n}\} \subset D \setminus \{a\}$, $x_{n} \rightarrow a$. Тогда
        \[f|_{D}(x_{n}) = f(x_{n}) \rightarrow b\]
        По определению Гейне, $b = \lim_{x \rightarrow a} f|_{D}(x)$.
    \end{proof}
    \item (о зажатой функции) Пусть $\exists \sigma > 0 \ \forall x \in \overset{o}{B_{\delta}} (a) \cap E \ (f(x) \leq h(x) \leq g(x))$. Пусть $\lim_{x \rightarrow a} f(x) = b$, $\lim_{x \rightarrow a} g(x) = b$. Тогда $\exists \lim_{x \rightarrow a} h(x) = b$.
    \begin{proof}
        Рассмотрим $x_{n} \subset E \setminus \{a\}$, $x_{n} \rightarrow a$. Тогда $\exists n_{0} \ \forall n \geq n_{0} (x_{n} \in \overset{o}{B_{\delta}}(a) \cap E)$ и, значит, $f(x_{n}) \leq h(x_{n}) \leq g(x_{n})$. По условию $f(x_{n}) \rightarrow b$, $g(x_{n}) \rightarrow b$. Тогда, по свойству предела последовательности, $h(x_{n}) \rightarrow b \Rightarrow b = \lim_{x \rightarrow a} h(x)$.
    \end{proof}
    \item (арифметические опреации с пределами) Пусть $\lim_{x \rightarrow a} f(x) = b$, $\lim_{x \rightarrow a} g(x) = c$. Тогда справедливы следующие утверждения:
    \\
    1. $\lim_{x \rightarrow a} (f(x) \pm g(x)) = b \pm c$.
    \\
    2. $\lim_{x \rightarrow a} (f(x) \cdot g(x)) = b \cdot c$.
    \\
    3. Если $c \neq 0$ и $g(x) \neq 0 \ \forall x \in E$, то $\lim_{x \rightarrow a} (\frac{f(x)}{g(x)}) = \frac{b}{c}$.
    
    Заключение следует понимать так: если существует величина справа, то существует величина слева и они равны.
    \begin{proof}
        Рассмотрим произвольную последовательность $\{x_{n}\} \in E$ с условиями $x_{n} \to a$ и $x_{n} \neq a$. Тогда $f(x_{n}) \to b$ и $g(x_{n}) \to c$. По свойствам предела последовательности $f(x_{n}) \pm g(x_{n}) \to b \pm c$, $f(x_{n}) \cdot g(x_{n}) \to b \cdot c$, $\frac{f(x_{n})}{g(x_{n})} \to \frac{b}{c}$. Осталось воспользоваться определением предела по Гейне.
    \end{proof}
    \item (о локализации) Если $\exists \sigma > 0 \ \forall x \in \overset{o}{B_{\sigma}}(a) \cap E \ (f(x) = g(x))$ и $\lim_{x \rightarrow a} f(x) = b$, то $\exists \lim_{x \rightarrow a} g(x) = b$.
    \begin{proof}
        Если в определении Коши предел $f$ для $\epsilon > 0$ подходит $\delta > 0$, то в поределении Коши предел $g$ подходит $\delta' = min\{\delta, \sigma\}$.
    \end{proof}
    \item (о локализации ограниченности) Если $\exists \lim_{x \rightarrow a} f(x) \in \R$, то $\exists C > 0 \ \exists \delta > 0 \ \forall x \in \overset{o}{B_{\delta}}(a) \cap E \ (|f(x)| \leq C)$.
    \begin{proof}
        Пусть $\lim_{x \rightarrow a} f(x) = b$. Тогда $\exists \delta > 0 \ \forall x \in \overset{o}{B_{\delta}}(a) \cap E \ (b - 1 < f(x) < b + 1)$. Положим $c = |b| + 1$. Тогда $|f(x)| < c$.
    \end{proof}
    \item (О пределе композиции.) Пусть $E, D \subset \R$ и $f: E \longrightarrow D$ и $g: D \longrightarrow \R$, такие что $\lim_{x \rightarrow a} f(x) = b$ и $\lim_{y \rightarrow b} g(y) = c$. Пусть выполнено одно из двух условий: 
    \\
    1) $f(x) \neq b$ в некоторой проколотой окрестности множества $a$ или 
    \\
    2) $g(b) = c$. Тогда $\lim_{x \rightarrow a} g(f(x)) = c = \lim_{y \rightarrow b} g(y)$.
    \begin{proof}
        Зафиксируем $\epsilon > 0$. По определению предела
        \[\exists \sigma > 0 \ \forall y \in \overset{o}{B_{\sigma}}(b) \cap D \ (g(y) \in \overset{}{B_{\epsilon}}(c))\]
        \[\exists \delta > 0 \ \forall y \in \overset{o}{B_{\delta}}(a) \cap E \ (f(x) \in \overset{}{B_{\sigma}}(b))\]
        1) Уменьшая $\delta$, если необходимо, можно считать, что $f(x) \neq b$ на $\overset{o}{B_{\delta}}(a) \cap E$. Тогда $f(x) \in \overset{o}{B_{\sigma}}(b) \cap D$. Поэтому $g(f(x)) \in \overset{}{B_{\epsilon}}(c) \Rightarrow \lim_{x \rightarrow a} g(f(x)) = c$.
        \\
        2) Если $f(x) = b$ для некоторого $x \in \overset{}{B_{\delta}}(a)$, то $g(f(x)) = c \in \overset{}{B_{\epsilon}}(c)$. Поэтому $\forall x \in \overset{}{B_{\delta}}(a) \cap E \ (g(f(x)) \in \overset{}{B_{\epsilon}}(c)) \Rightarrow \lim_{x \rightarrow a} g(f(x)) = c$.
    \end{proof}
\end{enumerate}

\subsubsection{Критерий Коши существования предела функции}

\begin{theorem}
	$\exists \liml_{x \ra a} f(x) \in \R \lra \underbrace{\forall \eps > 0\ \exists \delta > 0\ |\ \forall x_1, x_2 \in \mc{U}_{\delta}(a)\ |f(x_1) - f(x_2)| < \eps}_{\text{\normalfont Условие Коши}}$
\end{theorem}

\begin{proof}
	Докажем необходимость: \\
	Из определения предела:
	\[
		\liml_{x \ra a} f(x) = A \in \R \lra \forall \eps > 0\ \exists \delta > 0\ |\ \forall x \in \mc{U}_{\delta}(a)\ |f(x) - A| < \frac{\eps}{2}
	\]
	По неравенству треугольника: $\forall x_1, x_2 \in \mc{U}_{\delta}(a)\ |f(x_1) - f(x_2)| \le |f(x_1) - A| + |A - f(x_2)| < \eps$
	
	Докажем достаточность: \\
	Рассмотрим $\forall \{x_n\} \subset X \bs \{a\} \such \liml_{n \to \infty} x_n = a$. Из определения предела:
	$$
		\liml_{n \to \infty} x_n = a \lra \forall \delta > 0\ \exists N \in \N \such \forall n > N\ x_n \in \mc{U}_{\delta}(a)
	$$
	Согласно этому утверждению и условию Коши, мы получаем
	$$
		\forall \eps > 0\ \exists N \in \N \such \forall n, m > N\ |f(x_n) - f(x_m)| < \eps
	$$
	Что в точности означает фундаментальность последовательности $f(x_n)$, то есть она сходящаяся.
\end{proof}

\subsubsection{Различные типы пределов}

\begin{definition}
	Пусть $f$ определена на $(a; b)\ |\ -\infty < a < b < +\infty$
	
	\textit{Левосторонним пределом} в точке $b$ называется $B \in \bar{\R} \cup \{\infty\}$ такое, что
	\begin{enumerate}
		\item $\forall \eps > 0\ \exists \delta > 0\ |\ \forall x,\ b - \delta < x < b\ \ f(x) \in U_{\eps}(B)$
		
		\item $\forall \{x_n\}_{n = 1}^\infty \subset (a; b),\ \liml_{n \to \infty} x_n = b\ \ \liml_{n \to \infty} f(x_n) = B$
	\end{enumerate}
	Обозначается как
	\[
		f(b - 0) := \liml_{x \to b-0} f(x) = B
	\]

	\textit{Правосторонним пределом} в точке $a$ называется $A \in \bar{\R} \cup \{\infty\}$ такое, что
	\begin{enumerate}
		\item $\forall \eps > 0\ \exists \delta > 0\ |\ \forall x,\ a < x < a + \delta\ \ f(x) \in U_{\eps}(A)$
		
		\item $\forall \{x_n\}_{n = 1}^\infty \subset (a; b),\ \liml_{n \to \infty} x_n = a\ \ \liml_{n \to \infty} f(x_n) = A$
	\end{enumerate}
	Обозначается как
	\[
	f(b + 0) := \liml_{x \to a+0} f(x) = A
	\]	
\end{definition}

\begin{definition}
	$(b - \delta; b)$ называется \textit{левосторонней} окрестностью точки $b$.
	
	$(a; a + \delta)$ называется \textit{правосторонней} окрестностью точки $a$.
\end{definition}

\begin{theorem} (Связь предела и односторонних пределов)
	Пусть $f$ ограничена в $U_{\delta}(a)$, $a \in \R$. Тогда
	$$
		\exists \liml_{x \to a} f(x) \lra \exists \liml_{x \to a-0} f(x) = \liml_{x \to a+0} f(x)
	$$
\end{theorem}

\begin{proof}
\begin{enumerate}
	\item Пусть $\exists \liml_{x \to a} f(x) = A$, тогда
	$$
		\forall \eps > 0\ \exists \delta > 0\ |\ \forall x,\ 0 < |x - a| < \delta,\ f(x) \in U_{\eps}(A)
	$$
	Отсюда следует, что $\forall x\ |\ a < x < a + \delta,\ f(x) \in U_{\eps}(A)$ и $\forall x\ |\ a - \delta < x < a,\ f(x) \in U_{\eps}(A)$, что равносильно утверждению справа.
	
	\item Пусть $\exists \liml_{x \to a-0} f(x) = \liml_{x \to a+0} f(x) = A$. Тогда
	\begin{align*}
		\forall \eps > 0\ \exists \delta_1 > 0 \such \forall x,\ a - \delta_1 < x < a\ \ f(x) \in U_{\eps}(A)
		\\
		\forall \eps > 0\ \exists \delta_2 > 0 \such \forall x,\ a < x < a + \delta_2\ \ f(x) \in U_{\eps}(A)
	\end{align*}
	Выберем $\delta := \min(\delta_1, \delta_2)$, получим
	\begin{align*}
		\delta_1 \ge \delta \Ra a - \delta_1 \le a - \delta
		\\
		\delta_2 \ge \delta \Ra a + \delta_2 \ge a + \delta
	\end{align*}
	Рассмотрим $\forall x \in \mc{U}_\delta(a)$:
	\begin{align*}
		a < x < a + \delta \Ra a < x < a + \delta_2
		\\
		a - \delta < x < a \Ra a - \delta_1 < x < a
	\end{align*}
	Любой из этих случаев ведёт к тому, что $f(x) \in U_{\eps}(A)$. А значит
	\[
		\forall \eps > 0 \exists \delta > 0 \such \forall x,\ x \in \mc{U}_{\delta}(a)\ \ f(x) \in U_{\eps}(A)
	\]
	Что равносильно левой стороне утверждения.
\end{enumerate}

\end{proof}

\subsubsection{Существование односторонних пределов у монотонной функции}

\begin{theorem} (Существование односторонних пределов монотонной функции)
	Если $f$ монотонна на $(a; b)$, $-\infty < a < b < +\infty$, то
	$$
		\exists \liml_{x \to a+0} f(x) \in \bar{\R},\ \liml_{x \to b-0} f(x) \in \bar{\R}
	$$
	причём если $f$ неубывающая, то
	$$
		\liml_{x \to a+0} f(x) = \inf\limits_{x \in (a; b)} f(x),\ \liml_{x \to b-0} f(x) = \sup\limits_{x \in (a; b)} f(x)
	$$
	если $f$ невозрастающая, то
	$$
		\liml_{x \to a+0} f(x) = \sup\limits_{x \in (a; b)} f(x),\ \liml_{x \to b-0} f(x) = \inf\limits_{x \in (a; b)} f(x)
	$$
\end{theorem}

\begin{proof}
	Пусть $f$ неубывающая. Положим $\sup\limits_{x \in (a; b)} f(x) := M$
	\begin{enumerate}
		\item $M = +\infty$. Тогда
		\[
			\forall \eps > 0\ \exists x_0 \in (a; b)\ |\ f(x_0) > \frac{\dse 1}{\dse \eps}
		\]
		Отсюда $\exists \delta := b - x_0 > 0\ |\ \forall x, b - \delta < x < b \Ra \frac{\dse 1}{\dse \eps} < f(x) \le f(x)$, то есть $f(x_0) \in U_{\eps}(+\infty)$. В итоге
		\[
			\forall \eps > 0\ \exists \delta > 0 \such \forall x,\ b - \delta < x < b\ \ f(x) \in U_{\eps}(M) \lra \liml_{x \to b-0} f(x) = +\infty = M
		\]
		
		\item $M < +\infty$. Тогда
		\[
			\forall \eps > 0\ \exists x_0 \in (a; b)\ |\ f(x_0) \in (M - \eps; M]
		\]
		Отсюда уже аналогично получим, что
		\[
			\forall \eps > 0\ \exists \delta > 0 \such \forall x,\ b - \delta < x < b\ f(x) \in U_{\eps}(M) \lra \liml_{x \to b-0} f(x) = M
		\]
		Если $a = -\infty$, то $\liml_{x \to -\infty} f(x)$ вместо $\liml_{x \to a+0} f(x)$
		
		Если $b = +\infty$, то $\liml_{x \to +\infty} f(x)$ вместо $\liml_{x \to b-0} f(x)$
		
		
	\end{enumerate}
\end{proof}
