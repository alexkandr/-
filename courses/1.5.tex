\subsection{Свойства функций, непрерывных на отрезке — ограниченность... }
\subsubsection{Ограниченность}
\subsubsection*{Непрерывность на множестве}

\begin{definition}
	Функция называется \textit{непрерывной на множестве} $X$, если 
	\[
		\forall x_0 \in X\ \ \left(\forall \{x_n\} \subset X, \liml_{x \to \infty} x_n = x_0 \Ra \liml_{n \to \infty} f(x_n) = f(x_0)\right)
	\]
	или по Коши
	\[
		\forall x_0 \in X\ \ \left(\forall \eps > 0\ \exists \delta > 0 \such \forall x \in X \cap U_\delta(x_0)\ \left|f(x) - f(x_0)\right| < \eps\right)
	\]
\end{definition}

\begin{note}
	Не стоит думать, что непрерывность на множестве - это непрерывность в каждой точке этого множества. Это не так. Как минимум потому, что мы не требуем определённость функции в некоторой окрестности точки из $X$.
\end{note}

\begin{theorem} (Первая теорема Вейерштрасса о непрерывных на отрезке функциях)
	Если $f$ непрерывна на $[a; b]$, то она ограничена на $[a; b]$
\end{theorem}

\begin{proof}
	Докажем от противного. Пусть $f$ - неограничена сверху (снизу аналогично). Это означает
	$$
		\forall \eps > 0\ \exists x_{\frac{1}{\eps}} \in [a; b]\ |\ f(x_{\frac{1}{\eps}}) > \frac{1}{\eps}
	$$
	Последовательно будем брать $\eps := 1, \frac{1}{2}, \dots, \frac{1}{n}, \dots$
	
	Получим $\{x_n\}_{n = 1}^\infty \subset [a; b],\ f(x_n) > n$. По теореме Больцано-Вейерштрасса
	$$
		\exists \{x_{n_k}\}_{k = 1}^\infty,\ \liml_{k \to \infty} x_{n_k} = x_0 \in [a; b]
	$$
	А из этого следует, что $\liml_{k \to \infty} f(x_{n_k}) = f(x_0)$, что неверно ($f(x_n) > n$).
\end{proof}

\subsubsection{Достижение точных верхней и нижней граней}
\begin{theorem} (Вторая теорема Вейерштрасса о непрерывных на отрезке функциях) \\
	Если $f$ непрерывна на $[a; b]$, то она достигает своих точных верхней и нижней граней. То есть
	$$
		\exists x', x'' \in [a; b] \such f(x') = \inf\limits_{x \in [a; b]} f(x),\ f(x'') = \sup\limits_{x \in [a; b]} f(x)
	$$
\end{theorem}

\begin{proof}
	По определению минимума
	$$
		m := \inf\limits_{x \in [a; b]} f(x) \Ra \forall \eps > 0\ \exists x \in [a; b] \such m \le f(x) < m + \eps
	$$
	Построим подпоследовательность через выбор $\eps$:
	\begin{align*}
		&\eps := 1 & &m \le f(x_1) < m + 1
		\\
		&\eps := 1/2 & &m \le f(x_2) < m + 1/2
		\\
		&\dots & &\dots
		\\
		&\eps := 1/n & &m \le f(x_n) < m + 1/n
		\\
		&\dots & &\dots
	\end{align*}
	Получили ограниченную последовательность $\{x_n\}_{n = 1}^\infty$. По теореме Больцано-Вейерштрасса:
	$$
		\exists \{x_{n_k}\}_{k = 1}^\infty \subset \{x_n\}_{n = 1}^\infty \such \liml_{k \to \infty} x_{n_k} = x' \in [a; b] 
	$$
	Так как для $\forall n \in \N$ верно
	$$
		m \le f(x_n) < m + 1/n
	$$
	то в силу непрерывности $f$, можно совершить предельный переход:
	$$
		m \le f(x') \le m \lra f(x') = m
	$$
\end{proof}

\subsubsection{Теорема о промежуточных значениях непрерывной функции}

\begin{theorem} (Больцано-Коши о промежуточных значениях) \\
	Если $f$ непрерывна на $[a; b]$, то $\forall c = f(x_1) < d = f(x_2)$, где $\{x_1, x_2\} \subset [a; b]$ $\forall u \in (c; d)\ \exists \gamma \in [a; b] \such f(\gamma) = u$
\end{theorem}

\begin{anote}
	В классической версии данной теоремы утверждается, что $\exists \gamma$ не просто в $[a; b]$, а в $[\min(x_1, x_2); \max(x_1, x_2)]$. Из доказательства лектора это следует.
\end{anote}

\begin{proof}
	Рассмотрим $c < u = 0 < d$. Положим $\{a_1, b_1\} := \{x_1, x_2\}$. Не умаляя общности будем считать $a_1 < b_1$. В силу условия имеем
	\[
		f(a_1) \cdot f(b_1) < 0
	\]
	Посмотрим на $f(\frac{a_1 + b_1}{2})$. Если оно равно 0, то мы нашли подходящее нам $\gamma := \frac{a_1 + b_1}{2}$. Иначе рассмотрим одну из половин $[a_2; b_2]$ изначального отрезка такую, что на её концах функция тоже принимает разные значения (то есть \(\{\frac{a_1 + b_1}{2}\} \subset \{a_2, b_2\}\))
	\[
		f(a_2) \cdot f(b_2) < 0
	\]
	Продолжим рассуждения рекурсивно. Если мы так и не дошли до конкретного $\gamma$, то мы получили систему вложенных отрезков $\{[a_n; b_n]\}_{n = 1}^\infty$. Несложно заметить, что
	\begin{align*}
		&{[a_n; b_n] \supset [a_{n + 1}; b_{n + 1}]}
		\\
		&{b_n - a_n = \frac{b_1 - a_1}{2^{n - 1}}}
	\end{align*}
	То есть полученная система - стягивающаяся. А по принципу Кантора вложенных отрезков это нам даёт
	\[
		\exists \{\gamma\} = \bigcap\limits_{n = 1}^\infty [a_n; b_n]
	\]
	Равенство имеет место, потому что $\liml_{n \to \infty} a_n = \liml_{n \to \infty} b_n = \gamma$ по построению.\\
	В силу непрерывности функции и принципа Кантора
	\[
		\forall n \in \N\ a_n \le \gamma \le b_n \Ra f(\gamma) = \liml_{n \to \infty} a_n = \liml_{n \to \infty} b_n
	\]
	Так как по построению $f(a_n) \cdot f(b_n) < 0$, то предельный переход даёт неравенство
	\[
		f^2(\gamma) \le 0 \lra f(\gamma) = 0
	\]
	
	При любом другом $u$ мы можем рассмотреть вспомогательную функцию $F(x) = f(x) - u$, для которой верно доказанное утверждение, а значит получим и нужное:
	\[
		F(\gamma) = 0 = f(\gamma) - u \Ra f(\gamma) = u
	\]
\end{proof}

\subsubsection{Теорема об обратной функции.}

\begin{theorem} (Теорема об обратной функции)
	Если $f$ непрерывна и строго монотонна на промежутке $I$, то на промежутке $f(I)$ определена обратная функция $f^{-1}$, строго монотонная в том же смысле, что и $f$, и непрерывная на $f(I)$.
\end{theorem}

\begin{proof}
	Будем рассматривать такую $f$, что $\forall x_1 < x_2 \Ra f(x_1) < f(x_2)$. Положим
	\begin{align*}
		y_1 := f(x_1)
		\\
		y_2 := f(x_2)
	\end{align*}
	То есть
	\begin{align*}
		f^{-1}(y_1) := x_1
		\\
		f^{-1}(y_2) := x_2
	\end{align*}
	
	$f^{-1}(y_1) < f^{-1}(y_2)$, то есть $f^{-1}$ монотонно возрастает.
	
	По лемме \ref{for_back} $f(I)$ - промежуток. А значит, $f^{-1}$ определена на промежутке и при этом строго монотонна. Следовательно, по той же лемме, $f^{-1}$ - непрерывна на $f(I)$.
\end{proof}
