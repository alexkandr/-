\subsection{Непрерывность функции в точке. Свойства \dots} %непрерывных функций. Односторонняя непрерывность. Теорема о переходе к пределу под знаком непрерывной функции. Непрерывность сложной функции. Точки разрыва, их классификация. Разрывы монотонных функций.

\subsubsection{Непрерывность функции в точке. Односторонняя непрерывность.}

\begin{definition}
	Если $f$ определена в некоторой окрестности точки $x_0$ и $\liml_{x \to x_0} f(x) = f(x_0)$, то функция называется \textit{непрерывной} в точке $x_0$.
\end{definition}

\begin{definition}
	Если $f$ определена на $[x_0; x_0 + \delta_0]$, где $\delta_0 > 0$ и $\liml_{x \to x_0 + 0} f(x) = f(x_0)$, то $f$ называется \textit{непрерывной справа} в точке $x_0$.
\end{definition}

\begin{definition}
	Если $f$ определена на $[x_0 - \delta_0; x_0]$, где $\delta_0 > 0$ и $\liml_{x \to x_0 - 0} f(x) = f(x_0)$, то $f$ называется \textit{непрерывной слева} в точке $x_0$.
\end{definition}

\begin{theorem}
	Пусть $f$ определена в некоторой окрестности точки $x_0$. Тогда, следующие утверждения эквивалентны:
	\begin{enumerate}
		\item $f$ непрерывна в точке $x_0$
		\item $\forall \eps > 0\ \exists \delta > 0 \such \left(\forall x,\ |x - x_0| < \delta\right) \Ra |f(x) - f(x_0)| < \eps$
		\item $\left(\forall \{x_n\}_{n = 1}^\infty,\ \liml_{n \to \infty} x_n = x_0\right) \liml_{n \to \infty} f(x_n) = f(x_0)$
	\end{enumerate}
\end{theorem}

\subsubsection{Свойства непрерывных функций}

\subsubsection{Теорема о переходе к пределу под знаком непрерывной функции.}
\begin{theorem} (Переход к пределу под знаком непрерывной функции) \\
	Если $\liml_{x \to a} f(x) = b$ и $g$ непрерывна в точке $b$, то $\liml_{x \to a} (g \circ f)(x) = g(b)$
\end{theorem}

\begin{proof}
	Рассмотрим $\left(\forall \{x_n\}_{n = 1}^\infty,\ x_n \neq a,\ \liml_{n \to \infty} x_n = a\right)\ \liml_{n \to \infty} f(x_n) = b$
	
	Положим $y_n := f(x_n)$
	$$
		\{y_n\}_{n = 1}^\infty,\ \liml_{n \to \infty} y_n = b \Ra \liml_{n \to \infty} g(y_n) = g(b) 
	$$
\end{proof}

\subsubsection{Непрерывность сложной функции.}

\begin{addition} (Следствие теоремы выше. Непрерывность сложной функции)
	Если $f$ непрерывна в $a$, $g$ непрерывна в $f(a)$, то $g \circ f$ непрерывна в $a$.
\end{addition}

\begin{note} (Предел сложной функции)
	Для того, чтобы из $\liml_{x \to a} f(x) = b$ и $\liml_{y \to b} g(y) = l$ следовало $\liml_{x \to a} (g \circ f)(x) = l$, достаточно потребовать, чтобы $f(x) \neq b$ в некоторой проколотой окрестности точки $a$.
\end{note}

\subsubsection{Точки разрыва, их классификация.}

\begin{definition}
	Пусть $f$ определена в проколотой окрестности точки $x_0$. Если $\liml_{x \to x_0} f(x) \neq f(x_0)$, то $x_0$ называется \textit{точкой разрыва} функции $f(x)$.
\end{definition}

\begin{note}
	Неравенство полагается верным также и в тех случаях, когда хоть одна из частей не определена.
\end{note}

\begin{definition}
	Если $\exists \liml_{x \to x_0-0} f(x),\ \liml_{x \to x_0+0} f(x) \in \R$, то точка разрыва называется \textit{точкой разрыва первого рода}.
	
	В противном случае \textit{точкой разрыва второго рода}.
\end{definition}

\begin{definition}
	Если $\liml_{x \to x_0-0} f(x) = \liml_{x \to x_0+0} f(x) \in \R$ и $\neq f(x_0)$, то $x_0$ называется \textit{точкой устранимого разрыва}.
\end{definition}

\begin{definition}
	Если хотя бы 1 из односторонних пределов бесконечен, то $x_0$ называется \textit{точкой бесконечного разрыва}.
\end{definition}

\begin{definition}
	Величину $\liml_{x \to x_0+0} f(x) - \liml_{x \to x_0-0} f(x)$ называется \textit{скачком функции} в точке $x_0$.
\end{definition}

\subsubsection{Разрывы монотонных функций.}

\begin{theorem} (О точках разрыва монотонной функции)
	Если $f(x)$ монотонна на $(a; b),\ -\infty \le a < b \le +\infty$, то она может иметь на $(a; b)$ лишь точки разрыва 1го рода, причём неустранимого разрыва, и число таких точек разрыва не более чем счётно.
\end{theorem}

\begin{proof}
	$\forall x_0 \in (a; b)\ \exists \liml_{x \to x_0-0} f(x),\ \liml_{x \to x_0+0} f(x) \in \R$
	
	Считая $f$ неубывающей функцией, то 
	$$
		\forall x < x_0 \Ra f(x) \le f(x_0) \Ra f(x_0 - 0) \le f(x_0) \le f(x_0 + 0)
	$$
	$f(x_0 - 0) \neq f(x_0 + 0)$, иначе бы точки разрыва не было.
	
	$x_1 < x_2 \Ra f(x_1 - 0) < f(x_1 + 0) \le f(x_2 - 0) < f(x_2 + 0)$
	
	Отсюда $(f(x_1 - 0); f(x_1 + 0)) \cap (f(x_2 - 0); f(x_2 + 0)) = \emptyset$. То есть, мы получили систему непересекающихся отрезков на прямой действительных чисел, которая является не более чем счётным множеством (каждому отрезку можно сопоставить рациональное число внутри него).
\end{proof}

\begin{example} (Функция Римана)
	\[
		f(x) = \System{&{\frac{1}{n}, \text{ если } x = \frac{m}{n}} \\ &{0, \text{ если } x \in \R \bs \Q}}
	\]
	
	Докажем, что $f$ непрерывна в $x_0 \in \R \bs \Q$: зафиксируем произвольный $\eps > 0$ и рассмотрим множество
	\[
		M = \{x \such f(x) \ge \eps\}
	\]
	Так как $\eps > 0$ и $f(x) = 0\ \forall x \in \R \bs \Q$, то любой элемент $M$ - рациональное число, имеющее вид в несократимой дроби $\frac{m}{n}, m \in \Z, n \in \N$.
	\[
		f(x) = \frac{1}{n} \ge \eps \Ra n \le \frac{1}{\eps}
	\]
	То есть число таких $n$ конечно. Это значит, что число рациональных точек, попавших в $U_\delta(x_0) \cap M$, конечно (в самом деле, бесконечность может достигаться только за счёт $m$, а это мы ограничили пересечением). Ну а раз так, то найдётся $\delta > 0$ такое, что $U_\delta(x_0) \cap M = \emptyset$. Иными словами,
	\[
		\forall \eps > 0\ \exists \delta > 0 \such \forall x \in U_\delta(x_0)\ \ f(x) < \eps
	\]
	это означает непрерывность функции Римана в любой иррациональной точке.
	
	Теперь докажем, что $f(x)$ разрывна во всех рациональных точках. Пусть $x_0 \in \Q$ и мы снова зафиксировали $\eps > 0$. Какую $\delta$-окрестность точки $x_0$ ни взять, там найдётся иррациональное число, для которого $f(x) = 0 \Ra$ получим разрывность.
	
	Таким образом, функция Римана непрерывна $\forall x \in \R \bs \Q$ и разрывна $\forall x \in \Q$.
\end{example}